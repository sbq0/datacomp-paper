\section{Datasheet}

\begin{enumerate}[label=Q\arabic*]
\subsection{Motivation}

\item \textbf{For what purpose was the dataset created?} Was there a specific task in mind? Was there a specific gap that needed to be filled? Please provide a description.

\begin{itemize}
\item The purpose of \datanet and the associated \pool dataset is to enable study of what makes a strong image-text dataset, which supports a broad range of applications. Prior work mainly focuses on data curation in the context of supervised datasets and smaller scales. For a fuller treatment see Section~\ref{sec:relatedwork}. In our initial release of \datanet we focus on 38 downstream image classification and image retrieval tasks. We additionally explore two fairness datasets. For details see Section \ref{sec:evaluation} and Appendix \ref{sec:app-eval}.
\end{itemize}

\item \textbf{Who created the dataset (e.g., which team, research group) and on behalf of which entity (e.g., company, institution, organization)?}

\begin{itemize}
\item \datanet and \pool were created by a group of researchers with the following affiliations, listed in alphabetical order: Allen Institute for Artificial Intelligence (AI2), Apple, Columbia University, Graz University of Technology, Hebrew University, Juelich Supercomputing Center, LAION, Research Center Juelich, StabilityAI, Tel Aviv University, University of Illinois Urbana-Champaign, University of Texas at Austin, University of Washington.
\end{itemize}

\item \textbf{Who funded the creation of the dataset?} If there is an associated grant, please provide the name of the grantor and the grant name and number.

\begin{itemize}
\item Compute for this research was generously provided by StabilityAI. For more specific acknowledgments, see the acknowledgment section at the end of the main paper.
\end{itemize}

\item \textbf{Any other comments?}

\begin{itemize}
\item No.
\end{itemize}

\subsection{Composition}

\item \textbf{What do the instances that comprise the dataset represent (e.g., documents, photos, people, countries)?} \textit{Are there multiple types of instances (e.g., movies, users, and ratings; people and interactions between them; nodes and edges)? Please provide a description.}

\begin{itemize}
\item Each instance is a pair of url and corresponding image alt-text. The url points to an image that a user can then try to download. Each sample is also tagged with metadata, discussed in Q25.
\end{itemize}

\item \textbf{How many instances are there in total (of each type, if appropriate)?}

\begin{itemize}
\item There are 12.8B instances in \pool. For breakdowns and statistics see Appendix \ref{app:pool-stats}.
\end{itemize}

\item \textbf{Does the dataset contain all possible instances or is it a sample (not necessarily random) of instances from a larger set?} \textit{If the dataset is a sample, then what is the larger set? Is the sample representative of the larger set (e.g., geographic coverage)? If so, please describe how this representativeness was validated/verified. If it is not representative of the larger set, please describe why not (e.g., to cover a more diverse range of instances, because instances were withheld or unavailable).}

\begin{itemize}
\item We find $\sim$88B possible samples in common crawl. These samples are globally shuffled to ensure i.i.d. sampling for all sampling based parts of the downstream pipeline. Of these samples we attempt to download $\sim$40B samples. Due to various download issues, such as dead links and throttling, we are able to successfully download $\sim$16.8B samples. After NSFW filtering and evaluation set deduplication we end up with $\sim$13.1B viable samples, from which we randomly sample 12.8B for \pool. For a complete treatment and visualization of our data processing funnel, see Appendix \ref{app:metadata}. For each sample we also release metadata shown in Table \ref{tab:app_metadata}.
\end{itemize}

\item \textbf{What data does each instance consist of?} \textit{“Raw” data (e.g., unprocessed text or images) or features? In either case, please provide a description.}

\begin{itemize}
\item Each sample contains an image url for download and an associated alt-text caption. Additionally, each sample contains metadata fields shown in Table \ref{tab:app_metadata} (e.g., image aspect ratio and CLIP features).
\end{itemize}

\item \textbf{Is there a label or target associated with each instance?} \textit{If so, please provide a description.}

\begin{itemize}
\item We do not provide any category labels; however, the text associated with each image can be considered a soft, noisy label for each sample. Such labels are common in modern image-text training paradigms (e.g., image-text representation alignment, image captioning objectives, text-conditional image generation objectives, etc.).
\end{itemize}

\item \textbf{Is any information missing from individual instances?} \textit{If so, please provide a description, explaining why this information is missing (e.g., because it was unavailable). This does not include intentionally removed information, but might include, e.g., redacted text.}

\begin{itemize}
\item No.
\end{itemize}

\item \textbf{Are relationships between individual instances made explicit (e.g., users' movie ratings, social network links)?} \textit{If so, please describe how these relationships are made explicit.}

\begin{itemize}
\item No.
\end{itemize}

\item \textbf{Are there recommended data splits (e.g., training, development/validation, testing)?} \textit{If so, please provide a description of these splits, explaining the rationale behind them.}

\begin{itemize}
\item No. The test tasks are existing image classification tasks. We run a deduplication model to try to prevent test set contamination in \pool.
\end{itemize}

\item \textbf{Are there any errors, sources of noise, or redundancies in the dataset?} \textit{If so, please provide a description.}

\begin{itemize}
\item \pool is sourced from Common Crawl, which can be thought of as a snapshot of the internet. Hence, there can be considerable noise (e.g., alt-text being unrelated to its associated image), duplicate data, etc.
\end{itemize}

\item \textbf{Is the dataset self-contained, or does it link to or otherwise rely on external resources (e.g., websites, tweets, other datasets)?} \textit{If it links to or relies on external resources, a) are there guarantees that they will exist, and remain constant, over time; b) are there official archival versions of the complete dataset (i.e., including the external resources as they existed at the time the dataset was created); c) are there any restrictions (e.g., licenses, fees) associated with any of the external resources that might apply to a future user? Please provide descriptions of all external resources and any restrictions associated with them, as well as links or other access points, as appropriate.}

\begin{itemize}
\item The data is not self-contained and rather links other external resources on the internet. Links point to resources distributed across the internet. There is no guarantee that the resources will exist in perpetuity or that that the resources will not change. To mitigate against data poisoning in future \pool downloads, we release SHA256 hashes of images. Due to the size of the dataset, it is not possible to provide it in an archival form.
\end{itemize}

\item \textbf{Does the dataset contain data that might be considered confidential (e.g., data that is protected by legal privilege or by doctor–patient confidentiality, data that includes the content of individuals’ non-public communications)?} \textit{If so, please provide a description.}

\begin{itemize}
\item The dataset is comprised of data that was readily available on the internet at the time of our download. However, it is possible that the dataset contains confidential information (e.g., private data that is hosted publicly for nefarious reasons or out of ignorance of said data being confidential).
\end{itemize}

\item \textbf{Does the dataset contain data that, if viewed directly, might be offensive, insulting, threatening, or might otherwise cause anxiety?} \textit{If so, please describe why.}

\begin{itemize}
\item Considering the plurality of people and their backgrounds across the world, it is highly likely that there is content in \pool that may upset people. Common Crawl scrapes the internet, which has pornographic, hateful, racist, sexist, and otherwise abhorrent and toxic material. While we attempt to do thorough NSFW filtering, these methods are not 100\% accurate. At the 12.8B scale at which we operate it is highly likely that there is still toxic content in the dataset. We consider the dataset as a research artifact and hope future work will look critically at \pool in the hopes of developing even better safety filters.
\end{itemize}


\item \textbf{Does the dataset relate to people?} \textit{If not, you may skip the remaining questions in this section.}

\begin{itemize}
\item People may appear in the dataset; however, in an effort to preserve privacy, our downloading tooling automatically blurs all detected faces in \pool images.
\end{itemize}

\item \textbf{Does the dataset identify any subpopulations (e.g., by age, gender)?}

\begin{itemize}
\item While \pool does not explicitly identify subpopulations in its metadata, it is plausible to extract such information for some images using the corresponding textual caption.
\end{itemize}

\item \textbf{Is it possible to identify individuals (i.e., one or more natural persons), either directly or indirectly (i.e., in combination with other data) from the dataset?} \textit{If so, please describe how.}

\begin{itemize}
\item We conjecture that even with our face blurring procedure, it may still be possible to identify individuals. Face blurring relies of a face detection model, which could fail (See Appendix \ref{app:face} for experimental validation of the employed detector). It is also possible to identify certain celebrities or athletes, who may wear distinctive clothing that is associated with them. It is also likely that names are contained in textual captions, though it is not guaranteed that these names correspond to people in images due to the inherent noisiness of internet captions.
\end{itemize}

\item \textbf{Does the dataset contain data that might be considered sensitive in any way (e.g., data that reveals racial or ethnic origins, sexual orientations, religious beliefs, political opinions or union memberships, or locations; financial or health data; biometric or genetic data; forms of government identification, such as social security numbers; criminal history)?} \textit{If so, please provide a description.}

\begin{itemize}
\item Yes. \pool is created using images and corresponding alt-text that are available on the public internet. Given the 12.8B scale of \pool, it is highly likely that there is sensitive data in the dataset. To mitigate against making sensitive content more accessible, we 1) run NSFW image filtering and 2) NSFW text filtering when generating \pool, discarding all samples that are flagged. Additionally we 3) provide automatic face blurring in our \pool download scripts to blur all detected faces.
\end{itemize}

\item \textbf{Any other comments?}

\begin{itemize}
\item No.
\end{itemize}

\subsection{Collection Process}

\item \textbf{How was the data associated with each instance acquired?} \textit{Was the data directly observable (e.g., raw text, movie ratings), reported by subjects (e.g., survey responses), or indirectly inferred/derived from other data (e.g., part-of-speech tags, model-based guesses for age or language)? If data was reported by subjects or indirectly inferred/derived from other data, was the data validated/verified? If so, please describe how.}

\begin{itemize}
\item Data is directly downloaded from the public internet.
\end{itemize}

\item \textbf{What mechanisms or procedures were used to collect the data (e.g., hardware apparatus or sensor, manual human curation, software program, software API)?} \textit{How were these mechanisms or procedures validated?}

\begin{itemize}
\item We iterate on the LAION-5B data collection process, making an effort to emphasize safety. We ran python based processing scripts to parse Common Crawl dumps, download images, filter our NSFW content, deduplicate samples against downstream tests sets, blur faces, and compute CLIP features. We ran processes on 100s of AWS CPU nodes for Common Crawl parsing and data download. Other steps were run on one of StabilityAI's GPU cluster. For software links see Q37. For software validation related to NSFW content filtering and face blurring see Appendices \ref{app:nsfw} and \ref{app:face} respectively. In brief, for NSFW image filtering, we validate against commercial APIs and on the NSFW test set introduced in LAION-5B. For face detection (used for face blurring), we evaluate against commercial APIs and on the FairFace dataset. We find strong performance for both modules.
\end{itemize}

\item \textbf{If the dataset is a sample from a larger set, what was the sampling strategy (e.g., deterministic, probabilistic with specific sampling probabilities)?}

\begin{itemize}
\item See Q7.
\end{itemize}

\item \textbf{Who was involved in the data collection process (e.g., students, crowdworkers, contractors) and how were they compensated (e.g., how much were crowdworkers paid)?}

\begin{itemize}
\item The researching authors were involved in the data collection as an open source effort. No researchers were compensated specifically for their involvement in this project.
\end{itemize}

\item \textbf{Over what timeframe was the data collected? Does this timeframe match the creation timeframe of the data associated with the instances (e.g., recent crawl of old news articles)?} \textit{If not, please describe the timeframe in which the data associated with the instances was created.}

\begin{itemize}
\item Data was downloaded between December 2022 and March 2023. The urls are collected from Common Crawl dumps between 2014 and 2022. Common Crawl dumps may include urls from the early days of the internet. Hence, the download/collection timeframe does not match the creation timeframe. Additionally, future users of \pool and its subsets will have to download data themselves using our tooling.
\end{itemize}

\item \textbf{Were any ethical review processes conducted (e.g., by an institutional review board)?} \textit{If so, please provide a description of these review processes, including the outcomes, as well as a link or other access point to any supporting documentation.}

\begin{itemize}
\item Our dataset collection process iterates on the LAION-5B process, which found IRB review was not necessary as they ``do not intervene with the people depicted in the data as well as the data being public." Additionally, the NeurIPS ethics review found no serious ethical issues with LAION-5B. We take even more stringent safety measures than the original LAION-5B dataset, in that we filter out data that is flagged as NSFW by our detection pipeline and blur detected faces in \pool in our download scripts. All this being said, a formal ethics review has not been conducted to date.
\end{itemize}

\item \textbf{Does the dataset relate to people?} \textit{If not, you may skip the remaining questions in this section.}

\begin{itemize}
\item Yes. People may appear in the dataset. Detected faces are blurred when downloading \pool with our tooling.
\end{itemize}

\item \textbf{Did you collect the data from the individuals in question directly, or obtain it via third parties or other sources (e.g., websites)?}

\begin{itemize}
\item We collect data from websites across the internet.
\end{itemize}

\item \textbf{Were the individuals in question notified about the data collection?} \textit{If so, please describe (or show with screenshots or other information) how notice was provided, and provide a link or other access point to, or otherwise reproduce, the exact language of the notification itself.}

\begin{itemize}
\item Individuals were not notified about the data collection.
\end{itemize}

\item \textbf{Did the individuals in question consent to the collection and use of their data?} \textit{If so, please describe (or show with screenshots or other information) how consent was requested and provided, and provide a link or other access point to, or otherwise reproduce, the exact language to which the individuals consented.}

\begin{itemize}
\item Following our usage of Common Crawl and \url{https://github.com/rom1504/img2dataset} for download images, we respect \texttt{robots.txt} files, which specify parts of websites that a crawler may access. It is, however, possible that images of people, medical images, etc. were uploaded to the internet without a person's consent. To mitigate against such safety concerns we make an effort to do rigorous NSFW filtering and blur all detected faces automatically in our download tooling.
\end{itemize}

\item \textbf{If consent was obtained, were the consenting individuals provided with a mechanism to revoke their consent in the future or for certain uses?} \textit{If so, please provide a description, as well as a link or other access point to the mechanism (if appropriate).}

\begin{itemize}
\item In conjunction with LAION, we use \url{https://laion.ai/dataset-requests/} to monitor user takedown requests. We will also make an effort to provide a user with the url at which their sensitive content is hosted---if they do not have this information already---, so they can take further action as they see fit (e.g., contacting the host to request that the content is taken down from the internet).
\end{itemize}

\item \textbf{Has an analysis of the potential impact of the dataset and its use on data subjects (e.g., a data protection impact analysis) been conducted?} \textit{If so, please provide a description of this analysis, including the outcomes, as well as a link or other access point to any supporting documentation.}

\begin{itemize}
\item We conduct a fairness evaluation on models trained on \pool and its derivative. See Appendix \ref{app:fairness} for details. \citet{Birhane2021MultimodalDM} conduct an extensive study in the context of LAION-400M, which is an image-text dataset also sourced from Common Crawl, finding a plethora of dangerous and unsafe content. Our dataset differs from LAION-400M in that we conduct NSFW preprocessing and face blurring for detected faces. \pool only contains samples that pass our NSFW safety checks and our download tooling automatically blurs detected faces. However, since \pool is created from the internet, it is still likely that it contains some harmful data.
\end{itemize}

\item \textbf{Any other comments?}

\begin{itemize}
\item No.
\end{itemize}

\subsection{Preprocessing, Cleaning, and/or Labeling}

\item \textbf{Was any preprocessing/cleaning/labeling of the data done (e.g., discretization or bucketing, tokenization, part-of-speech tagging, SIFT feature extraction, removal of instances, processing of missing values)?} \textit{If so, please provide a description. If not, you may skip the remainder of the questions in this section.}

\begin{itemize}
\item Yes. See Q7.
For more details see Appendix \ref{app:metadata}.
\end{itemize}

\item \textbf{Was the “raw” data saved in addition to the preprocessed/cleaned/labeled data (e.g., to support unanticipated future uses)?} \textit{If so, please provide a link or other access point to the “raw” data.}

\begin{itemize}
\item Raw data is not available or distributed due to safety considerations. We distribute only urls that are in the dataset on HuggingFace---and not urls of images our preprocessing flagged as NSFW.
\end{itemize}

\item \textbf{Is the software used to preprocess/clean/label the instances available?} \textit{If so, please provide a link or other access point.}

\begin{itemize}
\item We use the following, open-source software to aid in data processing:
\begin{itemize}
\item Apache Spark: \url{https://spark.apache.org}
\item Ray: \url{https://www.ray.io}
\item img2dataset: \url{https://github.com/rom1504/img2dataset}
\item OpenAI CLIP: \url{https://github.com/openai/CLIP}
\item Near dedulicate detector: \url{https://github.com/lyakaap/ISC21-Descriptor-Track-1st}
\item Face detector: \url{https://github.com/deepinsight/insightface}
\item Detoxify, for detecting toxic language: \url{https://github.com/unitaryai/detoxify}
\item A modified version of the following NSFW image detector: 
\url{https://github.com/LAION-AI/CLIP-based-NSFW-Detector}. Specifically, we use the dataset used to train this model to train our own 4-layer MLP classifier.
\end{itemize}
\end{itemize}

\item \textbf{Any other comments?}

\begin{itemize}
\item No.
\end{itemize}

\subsection{Uses}

\item \textbf{Has the dataset been used for any tasks already?} \textit{If so, please provide a description.}

\begin{itemize}
\item The full dataset (and subsets) have been used to train several CLIP models at various scales and compute budgets as presented in our main paper. We evaluate these models zero-shot on 38 downstream image classification and retrieval tasks. We additionally evaluate on 2 fairness datasets. See Section \ref{sec:evaluation} and Appendix \ref{sec:app-eval} for more details.
\end{itemize}

\item \textbf{Is there a repository that links to any or all papers or systems that use the dataset?} \textit{If so, please provide a link or other access point.}

\begin{itemize}
\item No. However, there is a leaderboard associated with \datanet. Interested parties can investigate the submissions and further study publications that make use of our data. See: \url{https://www.datacomp.ai/leaderboard.html}.
\end{itemize}

\item \textbf{What (other) tasks could the dataset be used for?}

\begin{itemize}
\item The dataset could also be used for training image captioning models and language-conditional image generation models. Note: generative image models trained on \pool are not expected to generate recognizable human faces as our download tooling automatically blurs detected faces. \pool could be used for sociological studies, for example, examining societal biases or to better understand what is on the public internet.
\end{itemize}

\item \textbf{Is there anything about the composition of the dataset or the way it was collected and preprocessed/cleaned/labeled that might impact future uses?} \textit{For example, is there anything that a future user might need to know to avoid uses that could result in unfair treatment of individuals or groups (e.g., stereotyping, quality of service issues) or other undesirable harms (e.g., financial harms, legal risks) If so, please provide a description. Is there anything a future user could do to mitigate these undesirable harms?}

\begin{itemize}
\item In our initial analysis of models trained on \pool and its subsets, we notice disproportionate misclassification rates for identifying Black women (see Section \ref{app:fairness} for more details on our fairness and bias evaluation). \pool and its derivatives are not intended for production ready products, including but not limited to those related to race, gender identity or expression, ethnicity, sexual orientation, age, socioeconomic status, disability, religion, national origin or creed. \pool is not suitable for any software that makes decisions involving people. \pool is collected from the internet and hence reflects many of the biases, unfairness, and stereotypes currently existing in our societies. 
\end{itemize}

\item \textbf{Are there tasks for which the dataset should not be used?} \textit{If so, please provide a description.}
\begin{itemize}
\item \pool in its current form or the subsets presented in this paper should not be used in software that makes decisions related to people. The known biases make deploying software, especially widely decimated production-level products, built on \pool incredibly irresponsible. \pool is designed as a research artifact for academic exploration. We also do not condone the use of \pool in surveillance or military applications.
\end{itemize}

\item \textbf{Any other comments?}

\begin{itemize}
\item No.
\end{itemize}

\subsection{Distribution}

\item \textbf{Will the dataset be distributed to third parties outside of the entity (e.g., company, institution, organization) on behalf of which the dataset was created?} \textit{If so, please provide a description.}

\begin{itemize}
\item Yes. We use HuggingFace datasets for public release.
\end{itemize}

\item \textbf{How will the dataset be distributed (e.g., tarball on website, API, GitHub)?} \textit{Does the dataset have a digital object identifier (DOI)?}

\begin{itemize}
\item The dataset will be distributed via HuggingFace datasets at \href{https://huggingface.co/datasets/mlfoundations/datacomp_pools/tree/main}{\url{https://huggingface.co/datasets/mlfoundations/datacomp_pools/tree/main}}
\end{itemize}

\item \textbf{When will the dataset be distributed?}

\begin{itemize}
\item \datanet will be available starting May 2023.
\end{itemize}

\item \textbf{Will the dataset be distributed under a copyright or other intellectual property (IP) license, and/or under applicable terms of use (ToU)?} \textit{If so, please describe this license and/or ToU, and provide a link or other access point to, or otherwise reproduce, any relevant licensing terms or ToU, as well as any fees associated with these restrictions.}

\begin{itemize}
\item We distribute the url-text sample and metadata under a standard CC-BY-4.0 licence.
\end{itemize}

\item \textbf{Have any third parties imposed IP-based or other restrictions on the data associated with the instances?} \textit{If so, please describe these restrictions, and provide a link or other access point to, or otherwise reproduce, any relevant licensing terms, as well as any fees associated with these restrictions.}

\begin{itemize}
\item We do not copyright samples in the dataset.
\end{itemize}

\item \textbf{Do any export controls or other regulatory restrictions apply to the dataset or to individual instances?} \textit{If so, please describe these restrictions, and provide a link or other access point to, or otherwise reproduce, any supporting documentation.}

\begin{itemize}
\item No.
\end{itemize}

\item \textbf{Any other comments?}

\begin{itemize}
\item No.
\end{itemize}

\subsection{Maintenance}

\item \textbf{Who will be supporting/hosting/maintaining the dataset?}

\begin{itemize}
\item HuggingFace currently hosts the url-text pairs and metadata. The \datanet team will be responsible for maintaining the dataset.
\end{itemize}

\item \textbf{How can the owner/curator/manager of the dataset be contacted (e.g., email address)?}

\begin{itemize}
\item We can be contacted at \url{contact@datacomp.ai}.
\end{itemize}

\item \textbf{Is there an erratum?} \textit{If so, please provide a link or other access point.}

\begin{itemize}
\item Currently there are no errata. If issues are discovered, we will communicate with the public via our website \url{https://datacomp.ai}.
\end{itemize}

\item \textbf{Will the dataset be updated (e.g., to correct labeling errors, add new instances, delete instances)?} \textit{If so, please describe how often, by whom, and how updates will be communicated to users (e.g., mailing list, GitHub)?}

\begin{itemize}
\item At the present time there is no intention to update \pool for scientific reasons. However, we will respond to user takedown requests (see Q56). \pool is inherently noisy and the purpose of releasing it is to encourage researchers in the community to study dataset cleaning in the context of image-text samples.
\end{itemize}

\item \textbf{If the dataset relates to people, are there applicable limits on the retention of the data associated with the instances (e.g., were individuals in question told that their data would be retained for a fixed period of time and then deleted)?} \textit{If so, please describe these limits and explain how they will be enforced.}

\begin{itemize}
\item We will use the following website, \url{ https://laion.ai/dataset-requests}, for user takedown requests, where ``Sample ID'' is the sample uid.
\end{itemize}

\item \textbf{Will older versions of the dataset continue to be supported/hosted/maintained?} \textit{If so, please describe how. If not, please describe how its obsolescence will be communicated to users.}

\begin{itemize}
\item This is the first version of \datanet and the associated \pool dataset. We do not intend to maintain deprecated version of \pool. We will communicate deprication notices through our website: \url{https://datacomp.ai}.
\end{itemize}

\item \textbf{If others want to extend/augment/build on/contribute to the dataset, is there a mechanism for them to do so?} \textit{If so, please provide a description. Will these contributions be validated/verified? If so, please describe how. If not, why not? Is there a process for communicating/distributing these contributions to other users? If so, please provide a description.}

\begin{itemize}
\item All alterations to the dataset will be handled on a case-by-case basis.
\end{itemize}

\item \textbf{Any other comments?}

\begin{itemize}
\item No.
\end{itemize}

\end{enumerate}

